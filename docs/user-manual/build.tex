
\section{Building MCTrace}
\label{sec:building-mctrace}

MCTrace can be built for one of two purposes: either for local
development in a Haskell build environment or for release as a Docker
image. Instructions for each method are detailed below.

\subsection{Development Build Instructions}
\label{sec:devel-build-instr}

The development environment setup and build processes are automated.
The build process requires Ubuntu 20.04. To perform a one-time setup of
the development environment including the installation of LLVM, cross
compilers, and other required tools, run the development setup script:

\begin{verbatim}
./dev_setup.sh
\end{verbatim}

Once the development environment is set up and the required tools are
installed, MCTrace can be built with the build script:

\begin{verbatim}
./build.sh
\end{verbatim}

After the build has completed, various cross compilers and other tools
can be added to the \texttt{PATH} environment variable for easier access with:

\begin{verbatim}
. env.sh
\end{verbatim}

To build the example test programs for x86\_64 and instrument them
using various testing probes, run:

\begin{verbatim}
make -C mctrace/tests/full
\end{verbatim}

To do the same for PowerPC, run:

\begin{verbatim}
make -C mctrace/tests/full ARCH=PPC
\end{verbatim}

MCTRACE can then be run manually:

\begin{verbatim}
cabal run mctrace <args>
\end{verbatim}

\subsection{Release Build Instructions}
\label{sec:rele-build-instr}

To build the stand-alone release Docker image, execute the following from the root
of the repository:

\begin{verbatim}
cd release
./build.sh
\end{verbatim}

This will build two docker images: 
\begin{itemize}
\item A self-contained image that contains MCTrace, its dependencies,
  associated tools, and examples. For information on using this image,
  see Section~\ref{sec:user-release}.

\item A minimal image containing just MCTrace and its dependencies. A
  helper script, \texttt{release/mctrace} has been provided to run the
  command in a container. Note that paths passed to this script should
  be relative to the root of the repository and paths outside of the
  repository will not accessible. 
\end{itemize}

%%% Local Variables:
%%% mode: latex
%%% TeX-master: "user-manual"
%%% End:
