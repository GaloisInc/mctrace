DTrace is a popular tool supporting the definition of small snippets
of code which are then inserted into an application at specified
locations.\footnote{\url{https://en.wikipedia.org/wiki/DTrace}} For
example, these code snippets might be inserted directly before and
after every call made to a given function, for example a syscall such
as \texttt{write()}, in order to track the time elapsed each time the
function is called. Such probes are defined in the DTrace scripting
language and inserted into the source, which is then compiled.

The MCTrace tool is intended to provide the same functionality as DTrace
without requiring access to application source code or to the DTrace
run-time. For MCTrace, the probes defined as DTrace scripts are compiled
into the appropriate machine language for the target architecture and
are then inserted into the binary. Note that the binary must be modified
both to include the code for the probes themselves, as well as inserting
the calls to those probes at specified locations in the original binary
code.

\vspace{0.25in}


%%% Local Variables:
%%% mode: latex
%%% TeX-master: "user-manual"
%%% End:

