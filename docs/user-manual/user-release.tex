\section{User Release}
\label{sec:user-release}

This section describes how to run the standalone Docker image for
MCTrace described in Section~\ref{sec:introduction}.   Much of this
information is also available in the repository as
\texttt{release/README.md}, or in the standalone Docker image itself, as
\texttt{README.md}, at the top level.

To load and run the image, run the following commands from a directory
containing the image, using the filename \texttt{mctrace.tar.gz}

\begin{verbatim}
docker image load -i mctrace.tar.gz
docker run -it -w /mctrace-test mctrace
\end{verbatim}

This will drop you into a bash shell within the Docker container in the
directory \texttt{mctrace-test} where you can use \texttt{mctrace} to instrument
binaries.  All relative paths mentioned in this section are relative to
\texttt{mctrace-test}.

\subsection{Docker Image Contents}

The docker image contains PowerPC and x86\_64 test programs and example
probes that can be used to exercise MCTrace.  Important folders are as
follows:
\begin{itemize}
\item \texttt{examples/eval} contains a collection of probes primarily derived
   from those provided to us by WebSensing. The probes have been
   modified slightly after discussions with WebSensing to fit the
  currently supported DTrace syntax in MCTrace.
\item \texttt{examples/full} contains source code and binaries for bundled test
   programs.
\item \texttt{examples/binaries} contains binaries from a statically compiled
   version of GNU coreutils for use with \texttt{mctrace}.
 \end{itemize}
 
\subsection{Using MCTrace in this demonstration}

The \texttt{mctrace} tool is in the shell \texttt{PATH} when running within Docker,
so no special steps are needed to be able to invoke the executable.
For instructions on how to run the \texttt{mctrace} tool with the appropriate
command-line arguments, see \texttt{MCTRACE.md} included in the Docker image.

As an example, the \texttt{read-write-syscall-PPC.4.inst} binary in this
distribution is the instrumented version of the PowerPC binary
\texttt{read-write-syscall-PPC} and was instrumented with the following
\texttt{mctrace} command:

\begin{verbatim}
mctrace instrument --binary=/mctrace-test/examples/full/read-write-syscall-PPC \
   --output=/mctrace-test/examples/full/read-write-syscall-PPC.4.inst \
   --library=/mctrace-test/examples/library/PPC/platform_impl.o \
   --var-mapping=/mctrace-test/examples/full/read-write-syscall-PPC.4.json \
   --script=/mctrace-test/examples/eval/write-timing-probe.d
\end{verbatim}

\begin{itemize}
\item The \texttt{--binary} and the \texttt{--script} options tell mctrace to instrument
  the specified binary with the given probe script.
\item The \texttt{--output} option specifies the name for the instrumented binary.
\item The \texttt{--library} option specifies the path to the Platform API
  implementation.
\item The \texttt{--var-mapping} option tells \texttt{mctrace} where to record metadata
  that allows it to later interpret the collected telemetry.
\end{itemize}

The above command instruments the binary with probes that triggers
at the start and end of the \texttt{write} function and computes timing
information for the call. Note that the instrumentation command produces
a significant amount of DEBUG logs, that can be ignored at the moment.

When probes call the DTrace \texttt{send} action, the current test
implementation of \texttt{send} pushes the set of telemetry variables, in a
compact binary format, to the standard error. A script \texttt{extractor.py}
has been included with the image to help interpret this data.

To invoke the instrumented binary and use the \texttt{extractor.py} script to
decode any emitted telemetry:
\begin{verbatim}
/mctrace-test/examples/full/read-write-syscall-PPC.4.inst 2>&1
>/dev/null | \
  extractor.py /mctrace-test/examples/full/read-write-syscall-PPC.4.json \
  --extract --big-endian
\end{verbatim}
This produces output similar to the following:
\begin{verbatim}
{"write_count":1,"write_elapsed":162240,"write_ts":1681222607714740774} 
{"write_count":2,"write_elapsed":1740,"write_ts":1681222607714800756} 
{"write_count":3,"write_elapsed":1309,"write_ts":1681222607714803400} 
{"write_count":4,"write_elapsed":1344,"write_ts":1681222607714805742} 
{"write_count":5,"write_elapsed":1228,"write_ts":1681222607714807992} 
{"write_count":6,"write_elapsed":1222,"write_ts":1681222607714810214} 
{"write_count":7,"write_elapsed":1257,"write_ts":1681222607714812555} 
\end{verbatim}

Note that \texttt{2>\&1 >/dev/null} has the effect of piping the standard
  error to the next command while suppressing the standard output of the
  command. We do this because the provided platform API implementations
  writes \texttt{send()} data to \texttt{stderr} and we need that data to be piped to
  the extractor script.

When extracting telemetry data from instrumented PowerPC binaries, the
  flag \texttt{--big-endian} must be passed to the extractor script as in the
  command above. The flag should be elided when working with x86\_64
  binaries.

The \texttt{extractor.py} script offers a few other conveniences when
  extracting data from instrumented programs; for example it can produce
  columnar outputs and filter columns. See \texttt{extractor.py --help} for
  details on these options.  See 

See the file README.md for a list of other binaries for PowerPC and
x86\_64 included 
  in the Docker image as well some example probes that can be used to
  instrument each binary. Note that many other combinations of example
  programs and probes can work together; the full list of combinations
  can be found in \texttt{examples/full/Makefile}.
  
% \begin{center}
%   \begin{tabular}{ |l|l| }
%     \hline
%  Binaries &   Probe \\
% \hline 
% \hline 
%  \texttt{examples/full/alloc-dealloc-fread-fwrite-PPC},  \texttt{examples/full/alloc-dealloc-fread-fwrite-X86} & \texttt{examples/eval/fopen-calloc-fclose-probe.d} \\
% \hline 
% \texttt{examples/full/slow-read-write-PPC},  \texttt{examples/full/slow-read-write-X86} &                   \texttt{examples/eval/write-timing-probe.d}  \\
% \hline 
% \texttt{examples/full/read-write-syscall-PPC}, \texttt{examples/full/read-write-syscall-X86}                 & \texttt{examples/eval/graph-probe.d}   \\
% \hline 
%     \texttt{examples/full/array-sum-PPC}            &
%                                                       \texttt{examples/eval/copy-probe.d} \\
% \hline 
% \texttt{examples/binaries/PPC/cat}, \texttt{examples/binaries/X86/cat}                                     & \texttt{examples/eval/cat-probe.d} \\
% \hline 
% \texttt{examples/binaries/PPC/sha256sum}, \texttt{examples/binaries/X86/sha256sum}                           & \texttt{examples/eval/sha256sum-probe.d} \\
% \hline 
% \hline 
%   \end{tabular}
% \end{center}
% The first two probes above measure timing across different calls,
%   while the third one instruments *all* functions in the binary.

%%% Local Variables:
%%% mode: latex
%%% TeX-master: "user-manual"
%%% End:
  